
\documentclass[conference]{IEEEtran}
\IEEEoverridecommandlockouts

\usepackage{cite}
\usepackage{amsmath,amssymb,amsfonts}
\usepackage{graphicx}
\usepackage{textcomp}
\usepackage{xcolor}
\usepackage{hyperref}
\usepackage{etoolbox}

\def\BibTeX{{\rm B\kern-.05em{\sc i\kern-.025em b}\kern-.08em
    T\kern-.1667em\lower.7ex\hbox{E}\kern-.125emX}} 

\begin{document}

\title{\textbf{ML Assignment: Work Plan}\\
\vspace{10pt}
\Large Data Mining, University of Aveiro \\ 2019
}

\author{\IEEEauthorblockN{Filipe Pires, 85122}
\IEEEauthorblockA{\textit{DETI, MSc. Informatics Engineering}}
\and
\IEEEauthorblockN{João Alegria, 85048}
\IEEEauthorblockA{\textit{DETI, MSc. Informatics Engineering}}

}

\makeatletter
\patchcmd{\@maketitle}
{\addvspace{0.5\baselineskip}\egroup}
{\addvspace{-1\baselineskip}\egroup}
{}
{}
\makeatother

\maketitle

\section{Introduction}

The quest for better understanding how our brains work is always an intriguing
subject, so both assignment datasets were found interesting by us.
However, the Raven test, presented on the assignment instructions \cite{assign2},
caught our eyes as it aimed at analyzing intelligent measures in the general 
population.

For this reason, and taking in consideration that the participants had background
knowledge similar to ours and to people we deal with on a daily basis, we chose 
the second dataset to analyse using a supervised machine learning algorithm.

In this document we present our work plan for the assignment, including the 
chosen algorithm, the question to be addressed and the way we are going to 
organize our time.

\section{Dataset}

According to the assignment's description, the dataset to be used corresponds to
the application of the Advanced Progressive Matrices test with 48 problems to 2
distinct populations: the 1st consisting of 21 students of Design and Multimedia
(DM); the 2nd of 24 students of Informatics Engineering (IE).

The data was collected from EEG signal receptors while the participants 
performed the test tasks.
The registered signals were from 8 channels: F3, F4, T7, C3, Cz, C4, T8 and Pz.
The relevant marks for the signal analysis are presented in Table 1, along with 
the respective signal processing windows.

\begin{table}[h!]
    \begin{center}
        \begin{tabular}{c|c} %
            \textbf{Time Marks} & \textbf{Signal Windows (ms)}\\
            \hline
            Problem Display & [-75 500] \\
            Possible Solutions Display & [-75 500] \\
            Student Answer & [-500 500] \\
        \end{tabular}
    \end{center}
    \caption{}
\end{table}
\vspace{-15pt}

The time features extracted from the raw data were \textbf{P100} and \textbf{P300}, 
due to their significant characteristics for the study.
Also, through the energy (E) of the characteristic bands estimated in all defined
windows, some frequency features defined as energy ratios are included:
\textbf{Stress Index; Mental Fatigue; Alpha Lateralization; Immersion Index}.

Having gathered this knowledge and analysed the available dataset and its structure
and division between train and test sets or average and non-average results, we
were able to discuss how could the analysis be done and what algorithms might
be useful for the proposed machine learning tasks.

\newpage
\section{Goals \& Strategies}

The first goal of our work is to understand in greater depth the dataset, its 
quality and normalization, its features, and search for outliers and the 
percentage of non-successful measurements on the trials.
Once a refined dataset is achieved, we intend to address the more demanding 
question of whether there is a significant difference between the 2 groups of 
participants or not, and, if so, how might this difference be characterized.

With this in mind, and considering the fact that the solution should address 
supervised algorithms only, we decided to apply the following machine learning 
algorithms to the dataset to seek for answers to our posed questions:
\textbf{Neural Network} and \textbf{Support Vector Machine}.
We chose these algorithms since we are facing a classification problem, where 
the possible classes are the study field of the participant whose features are 
used as input.
SVMs and NNs are famous for their high performance in issues such as these, but
we are considering exploring additional promising alternative worth testing.

\section{Work Plan}

The first phase of development will be to analyse the available features 
(extracted in different ways) and use only the data considered relevant.
The resulting dataset is to be divided into 3 subsets: train, validation and test.

Then, we intend to develop the algorithms in Python, with the help of libraries
dedicated to statistical analysis and data science, and train the models with the
proper subset.
A study is to be conducted on hyperparameters variation and the models are to be
refined with the knowledge gained from this.

At this phase we will compare the algorithms in terms of performance, using 
several appropriate metrics.
Simultaneously we will also attempt to answer the question "are the two groups
different?" based on the results.

The work is to be fairly distributed between team elements and we intend 
to schedule dedicated time for the project in compatible hours so that more 
important issues can be addressed by the whole team.


% \begin{itemize}
% \item Lorem ipsum ...
% \end{itemize}

% \begin{equation}
% a+b=\gamma\label{eq}
% \end{equation}

% ... \eqref{eq} ...

% ... Fig.~\ref{fig} ...

% \begin{figure}[h!]
% \centering
% \includegraphics[width=0.95\linewidth]{../img/screenshots/screenshot_map.png}
% \caption{Screen capture of our prototype's world map.}
% \label{fig:worldmap}
% \end{figure}

% \begin{figure}[htbp]
% \centerline{\includegraphics{fig1.png}}
% \caption{Example of a figure caption.}
% \label{fig}
% \end{figure}

\begin{thebibliography}{00}
\bibitem{assign2} A. Tomé, "Data Mining Assignment", \url{https://elearning.ua.pt/pluginfile.php/1496406/mod_resource/content/3/ED_HCT_Raven.pdf}, accessed in November 2019.
\end{thebibliography}
\vspace{12pt}

\end{document}


\documentclass[conference]{IEEEtran}
\IEEEoverridecommandlockouts

\usepackage{cite}
\usepackage{amsmath,amssymb,amsfonts}
\usepackage{graphicx}
\usepackage{textcomp}
\usepackage{xcolor}
\usepackage{hyperref}
\usepackage{etoolbox}

\def\BibTeX{{\rm B\kern-.05em{\sc i\kern-.025em b}\kern-.08em
    T\kern-.1667em\lower.7ex\hbox{E}\kern-.125emX}} 

\begin{document}

\title{\textbf{Assignment 2: The Raven Test \\ \LARGE{In Search for Group Differences}}\\
\vspace{10pt}
\Large Data Mining, University of Aveiro \\ 2019
}

\author{\IEEEauthorblockN{Filipe Pires, 85122}
\IEEEauthorblockA{\textit{DETI, MSc. Informatics Engineering}}
\and
\IEEEauthorblockN{João Alegria, 85048}
\IEEEauthorblockA{\textit{DETI, MSc. Informatics Engineering}}

}

\makeatletter
\patchcmd{\@maketitle}
{\addvspace{0.5\baselineskip}\egroup}
{\addvspace{-1\baselineskip}\egroup}
{}
{}
\makeatother

\maketitle

\begin{abstract}
The search for trustworthy methodologies of determining a measurable intelligence
index has been a quest of our brightest minds for decades.
Raven matrices tests have proved to have widespread practical use as a measure 
of intelligence.
They are a source of data for many studies on the general population as they seem 
promising tools for contexts such as psychometric tests or clinical assessment.

In one study on the application of these matrices to groups of students from 
different backgrounds, questions emerged regarding the possibility of clear 
differences between Multimedia and Informatics students.
In this paper we present the statistical analysis applied to the tests results 
with the help of ML classification techniques in search for determining whether 
any of the two groups showed significant advantages over the other.
\end{abstract} 

\begin{IEEEkeywords}
Raven Matrices, Psychometric Tests, Intelligence Measure, 
Support Vector Machine, Multi-Layer Perceptron, Decision Tree, K-Nearest-Neighbors
\end{IEEEkeywords}

\section{Introduction}

Raven's Advanced Progressive Matrices (RAPM) is a non-verbal group test 
typically present in educational or clinical settings, as it is used in 
measuring abstract reasoning and regarded as a non-verbal estimate of fluid 
intelligence \cite{rapm}.
Examples of related test are Naglieri Nonverbal Ability or Spacial Ability Tests.
Their practical use is very extended, and applicable to both adults and children.
Nevertheless, studies that resort to them usually focus on populations containing 
groups with specific differences in order to draw conclusions from these differences.
Examples of these studies are on different military sections, or different 
mental disabilities.

In the study whose collected data was used for our analysis \cite{study}, the 
aim was to compare university students from different fields in terms of 
learning styles effectiveness.
Several tests such as Kolb and VAK or Hermann dominances allow to distinguish
some learning styles like: Accommodator, Assimilator, Auditory, Convergent,
Divergent, Kinesthetic and Visual.
But beyond this, the researchers also applied the RAPM tests to reach more 
robust conclusions, and combine all results in a meaningful way.
In this paper we focus only on the data related to the second set of tests.

The population that conducted the Raven tests was a group of 45 university 
students, 21 of Design and Multimedia and 24 of Informatics Engineering.
48 problems were presented to the distinct populations and were divided into 
two phases: during the first 12, the participant would receive a feedback about 
his/her answer; for the remaining 36 no feedback was given.
During the test execution electroencephalographic (EEG) signals were registered 
while the participants performed the tasks, using Enobio 8 EEG recording headset 
and 8 channels: \textit{F3, F4, T7, C3, Cz, C4, T8} and \textit{Pz}.

Our aim is to determine, solely from this estimated measurement of intelligence,
whether both groups hold characteristics significantly different from each other
by building classification algorithms that interpret the EEG signals and other 
time-related metrics as features and attempt to predict which class of students
a new entry belongs to.
We also intend to compare our conclusions with those obtained by the original researchers.

\section{Dataset \& Feature Extraction}

Lorem ipsum ...



\section{Data Quality \& Normalization}

Lorem ipsum ...

\section{Classifiers}

Lorem ipsum ...

\section{Parameters Variation}

Lorem ipsum ...

\section{Results Discussion}

Lorem ipsum ...

\section{Conclusions \& Future Work}

Lorem ipsum ...

% \begin{itemize}
% \item Lorem ipsum ...
% \end{itemize}

% \begin{equation}
% a+b=\gamma\label{eq}
% \end{equation}

% ... \eqref{eq} ...

% ... Fig.~\ref{fig} ...

% \begin{table}[htbp]
% \caption{Table Type Styles}
% \begin{center}
% \begin{tabular}{|c|c|c|c|}
% \hline
% \textbf{Table}&\multicolumn{3}{|c|}{\textbf{Table Column Head}} \\
% \cline{2-4} 
% \textbf{Head} & \textbf{\textit{Table column subhead}}& \textbf{\textit{Subhead}}& \textbf{\textit{Subhead}} \\
% \hline
% copy& More table copy$^{\mathrm{a}}$& &  \\
% \hline
% \multicolumn{4}{l}{$^{\mathrm{a}}$Sample of a Table footnote.}
% \end{tabular}
% \label{tab1}
% \end{center}
% \end{table}

% \begin{figure}[h!]
% \centering
% \includegraphics[width=0.95\linewidth]{../img/screenshots/screenshot_map.png}
% \caption{Screen capture of our prototype's world map.}
% \label{fig:worldmap}
% \end{figure}

% \begin{figure}[htbp]
% \centerline{\includegraphics{fig1.png}}
% \caption{Example of a figure caption.}
% \label{fig}
% \end{figure}

\begin{thebibliography}{00}
\bibitem{rapm} Warren B. Bilker et al., "Development of Abbreviated Nine-item Forms of the Raven’s Standard Progressive Matrices Test", \url{https://www.ncbi.nlm.nih.gov/pmc/articles/PMC4410094}, accessed in December 2019.
\bibitem{study} Felisa M. Córdova et al., "Identifying Problem Solving Strategies for Learning Styles in Engineering Students Subjected to Intelligence Test and EEG Monitoring", \url{https://www.sciencedirect.com/science/article/pii/S1877050915014787}, Procedia Computer Science 55 (2015), accessed in December 2019.
\bibitem{assign2} A. Tomé, "Data Mining Assignment", \url{https://elearning.ua.pt/pluginfile.php/1496406/mod_resource/content/3/ED_HCT_Raven.pdf}, accessed in December 2019.

\end{thebibliography}
\vspace{12pt}

\end{document}
